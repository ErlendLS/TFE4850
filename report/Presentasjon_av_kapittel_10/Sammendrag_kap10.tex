\documentclass[11pt]{article}

\usepackage[utf8]{inputenc}

\title{Sammendrag av kapittel 10 - Eksperter i team-kompendie}
\author{Jonas B. Ghini}

\begin{document}

\maketitle

	\section{Forord (Ja. Ja, jeg tar med et forord :-P)}
		Jeg har forsøkt å lage et sammendrag av det kapittelet det er meningen av vi skal presentere onsdag 12. mars. Siden jeg bare er et menneske er jeg redd teksten her blir farget av mine egne tanker og forutinntattheter, men jeg forsøker å gi en så objektiv gjengivelse som mulig. Alle de kommende "section"-titlene følger strukturen til artikkelen/kapittelet som vi skal presentere. Jeg tillater meg å komme med en grunnleggende observasjon om teksten i kompendiet: den er temmelig amerikansk: Amerika her og vi er best der og alle i denne store nasjonen kommer med sine kulturer, bla, bla, bla. Enda en sak: mange steder har jeg skrevet ordet "variasjon"; det var den oversettelsen jeg først kom på. Så kom jeg på "mangfold", som i denne mer sosiale sammenhengen er mer korrekt. Meeeen jeg gidder ikke rette på det. Der det står variasjon, mener jeg mangfold.
				
	\section{Introduksjon}
		Teksten åpner med å sammenligne en liten gruppe med mennesker (slik vi i EiT-gruppa er et eksempel på) med historien om "Skjønnheten og Udyret". Tanken er at innad i en liten gruppe vil det være stor variasjon mellom medlemmene\footnote{Jeg mener teksten er litt svak her: også i store grupper er det rom for variasjon. Jeg tror det de mener her er at variasjon mellom individer vil være mer fremtredende i en liten gruppe, da en større gruppe vil ha mulighet til å skape subkulturer og at det er større rom for forskjeller siden en har flere "normale" å fordele det "rare" på for fremdeles å oppnå det samme gjennomsnittet.}, og at en gruppe må både akseptere og ideelt sett sette pris på den variasjonen gruppemedlemmene har seg imellom.
		\newline \newline
		Hovedpoenget som presenteres i dette kapittelet er at denne variasjonen kan både styrke og svekke en gruppes evne til å yte godt og å fungere sammen. Eksempler på konsekvenser av variasjon:
		
		\begin{itemize}
			\item[\textbf{Positive}] Økt produktivitet og prestasjon; kreativ problemløsing; vekst/forbedring i kognitiv og moralsk\footnote{Oversetter litt verbatim her; kan hende tolkningen egentlig skulle være en annen} refleksjonsevne; bredere perspektiver; forbedrede forhold; og mer sofistikert interaksjon og samarbeid med sine medstudenter/kolleger.
			
			\item[\textbf{Negative}] Redusert produktivitet og prestasjon\footnote{De dragvoller oss litt opp etter veggen her: "en kan få denne effekten av dette, men man kan også få helt motsatt effekt av akkurat det samme!"}; inaksept for ny informasjon; økt egosentrisitet; økt fiendtlighet, splittelse, avvisning, hersing/bølling, fordømming, rasisme og uthenging/"svartepering"\footnote{Jada, jada... Scapegoating var det engelske ordet}.
		\end{itemize}
		
		Hvorvidt variasjonen fører med seg positive eller negative konsekvenser avhenger av gruppemedlemmenes villighet og evne til å forstå og verdsette denne variasjonen. Utfallet avhenger mer spesifikt av:
		
		\begin{enumerate}

			\item Anerkjenne at variasjonen eksisterer og at dette er en potensiell ressurs.

			\item Skape seg en sammenhengende personlig identitet som inkluderer (a) ens egen kulturelle/etniske arv, og (b) et syn på seg selv som et individ som respekterer andre individer\footnote{Det blandes litt mellom hva som gjelder gruppa som helhet og hva som gjelder gruppas medlemmer. Her tipper jeg det ikke er snakk om at gruppa skal skape seg en kollektiv identitet ala amerikansk høyesteretts syn på børsnoterte selskaper :-P}.
			
			\item Forstå de interne kognitive barrierene (som stereotypier og fordommer) en som individ har mot å skape nye forhold til sine like, og å jobbe for å redusere disse.
			
			\item Forstå dynamikken i intergruppekonflikter\footnote{På engelsk står det "inter group conflict". Jeg tipper de mener "intragruppekonflikt", da det neppe er så interessant for EiT hvordan de fem gruppene i samme landsby kommuniserer og fungerer med hverandre.}
			
			\item Forstå den sosiale dømmelsesprosessen, og å vite hvordan en skaper en positiv aksept-prosess samtidig som en hindrer en negativ avvisnings-prosess.
			
			\item Skape en gruppekontekst som legger til rette for samarbeid, ved å legge vekt på kollektivitet fremfor individualisme, slik at gruppen internt bygger personlige forhold fremfor upersonlige forhold.
			
			\item Konstruktiv konflikthåndtering hvor det legges særlig vekt på skillet mellom (a) intellektuelle konflikter knyttet til beslutninger og læring, og (b) interessekonflikter.
			
			\item Lære og internalisere fellesskapsorienterte og demokratiske verdier.
		\end{enumerate}
	
	\section{Kilder til variasjon}
		Tre store kilder til individuell variasjon er å finne i (1) demografisk karakteristikk, (2) personlig karakteristikk, (3) evner/kunnskaper. 
		\textbf{Demografisk variasjon} inkluderer kultur, etnisitet, språk, kjønn, funksjonshemninger, alder, sosial status og geografisk tilhørighet. Med \textbf{personlig karakteristikk} menes ting som økonomisk bakgrunn og kommunikasjonsstil\footnote{For min egen del tenker jeg en kan legge til: generelt humør, humor, interesser, seksuell legning, holding til andre, etc.}. Det nevnes også ting som introvert vs ekstrovert personlighet, en persons tilnærming til problemer; tilfeldig eller prosedyreorientert. Holdinger til innovasjon trekkes også frem her, knyttet til en persons utdannelsesnivå. \textbf{Evner og talenter}, av både sosial og teknisk karakter, er individuelt forskjellige blant alle i gruppen. Det er vanskelig å finne en fungerende/produktiv gruppe der det ikke er stor variasjon i evner og kunnskaper mellom medlemmene.
		
	\section{Viktigheten av å forvalte variasjon}
		Å utnytte variasjon på en slik måte at mange positive og få negative utfall følger, er en av det moderne samfunns største utfordringer. Gode måter å håndtere variasjon på er viktig av flere grunner. 
		\newline \newline
		For det første lever vi i stadig større grad i "en verden". Problemer individer, samfunn og nasjoner står overfor må løses med globalt samarbeid\footnote{Å, kom igjen, Amerika! At jeg mangler tannkrem er ikke en sak for FNs sikkerhetsråd!}. Jo mer gjensidig/internt avhenging verden blir, desto større variasjon innen en gruppe må en forvente. En model som bygger på dette kalles "global village".
		\newline \newline
		For det andre er denne variasjonen praktisk talt umulig å unngå. I den moderne tid er emigrasjon en ny faktor som før ikke på langt nær spilte den samme rollen. I tillegg er det nå mye større kommunikasjon på tvers av sosiale skiller enn det var før.
		\newline \newline
		For det tredje er globaliseringen av verdensøkonomien en stor bidragsyter til å forsterke variasjon: folk flytter på seg for å følge arbeid og karriere, og kommer med det inn i nye grupper der de selv er "variasjonen". 
		
		\begin{quote} Morgendagens effektive grupper (inklusive store grupper som organisasjoner og nasjoner) er de som har lært seg å være produktive med et stort mangfold i medlemskapet. \end{quote}
		
		Resten av dette kapittelet handler om hvordan grupper kan utnytte de positive sidene av mangfoldet, og minimere de negative.
		
	\section{Verdien av mangfold}
		\subsection{Gruppesammensetning og oppgaveutførelse}
			

\end{document}