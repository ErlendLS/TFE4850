\documentclass[11pt]{article}

\usepackage[utf8]{inputenc}

\title{Sammendrag av kapittel 10 - Eksperter i team-kompendie}
\author{Jonas B. Ghini}

\begin{document}

\maketitle

	\section{Forord (Ja. Ja, jeg tar med et forord :-P)}
		Jeg har forsøkt å lage et sammendrag av det kapittelet det er meningen av vi skal presentere onsdag 12. mars. Siden jeg bare er et menneske er jeg redd teksten her blir farget av mine egne tanker og forutinntattheter, men jeg forsøker å gi en så objektiv gjengivelse som mulig. Alle steder hvor jeg aktivt kommer med kommentarer eller tanker knyttet til teksten er skilt ut som fotnoter. Brødteksten er altså en så upåvirket oversettelse/sammendragning av det som står i kompendiet.
		
		 Alle de kommende "section"-titlene følger strukturen til artikkelen/kapittelet som vi skal presentere. Jeg tillater meg å komme med en grunnleggende observasjon om teksten i kompendiet: den er temmelig amerikansk: Amerika her og vi er best der og alle i denne store nasjonen kommer med sine kulturer, bla, bla, bla. Enda en sak: mange steder har jeg skrevet ordet "variasjon"; det var den oversettelsen jeg først kom på. Så kom jeg på "mangfold", som i denne mer sosiale sammenhengen er mer korrekt. Meeeen jeg gidder ikke rette på det. Der det står variasjon, mener jeg mangfold.
				
	\section{Introduksjon}
		Teksten åpner med å sammenligne en liten gruppe med mennesker (slik vi i EiT-gruppa er et eksempel på) med historien om "Skjønnheten og Udyret". Tanken er at innad i en liten gruppe vil det være stor variasjon mellom medlemmene\footnote{Jeg mener teksten er litt svak her: også i store grupper er det rom for variasjon. Jeg tror det de mener her er at variasjon mellom individer vil være mer fremtredende i en liten gruppe, da en større gruppe vil ha mulighet til å skape subkulturer og at det er større rom for forskjeller siden en har flere "normale" å fordele det "rare" på for fremdeles å oppnå det samme gjennomsnittet.}, og at en gruppe må både akseptere og ideelt sett sette pris på den variasjonen gruppemedlemmene har seg imellom.
		\newline \newline
		Hovedpoenget som presenteres i dette kapittelet er at denne variasjonen kan både styrke og svekke en gruppes evne til å yte godt og å fungere sammen. Eksempler på konsekvenser av variasjon:
		
		\begin{itemize}
			\item[\textbf{Positive}] Økt produktivitet og prestasjon; kreativ problemløsing; vekst/forbedring i kognitiv og moralsk\footnote{Oversetter litt verbatim her; kan hende tolkningen egentlig skulle være en annen} refleksjonsevne; bredere perspektiver; forbedrede forhold; og mer sofistikert interaksjon og samarbeid med sine medstudenter/kolleger.
			
			\item[\textbf{Negative}] Redusert produktivitet og prestasjon\footnote{De dragvoller oss litt opp etter veggen her: "en kan få denne effekten av dette, men man kan også få helt motsatt effekt av akkurat det samme!"}; inaksept for ny informasjon; økt egosentrisitet; økt fiendtlighet, splittelse, avvisning, hersing/bølling, fordømming, rasisme og uthenging/"svartepering"\footnote{Jada, jada... Scapegoating var det engelske ordet}.
		\end{itemize}
		
		Hvorvidt variasjonen fører med seg positive eller negative konsekvenser avhenger av gruppemedlemmenes villighet og evne til å forstå og verdsette denne variasjonen. Utfallet avhenger mer spesifikt av:
		
		\begin{enumerate}

			\item Anerkjenne at variasjonen eksisterer og at dette er en potensiell ressurs.

			\item Skape seg en sammenhengende personlig identitet som inkluderer (a) ens egen kulturelle/etniske arv, og (b) et syn på seg selv som et individ som respekterer andre individer\footnote{Det blandes litt mellom hva som gjelder gruppa som helhet og hva som gjelder gruppas medlemmer. Her tipper jeg det ikke er snakk om at gruppa skal skape seg en kollektiv identitet ala amerikansk høyesteretts syn på børsnoterte selskaper :-P}.
			
			\item Forstå de interne kognitive barrierene (som stereotypier og fordommer) en som individ har mot å skape nye forhold til sine like, og å jobbe for å redusere disse.
			
			\item Forstå dynamikken i intergruppekonflikter\footnote{På engelsk står det "inter group conflict". Jeg tipper de mener "intragruppekonflikt", da det neppe er så interessant for EiT hvordan de fem gruppene i samme landsby kommuniserer og fungerer med hverandre.}
			
			\item Forstå den sosiale dømmelsesprosessen, og å vite hvordan en skaper en positiv aksept-prosess samtidig som en hindrer en negativ avvisnings-prosess.
			
			\item Skape en gruppekontekst som legger til rette for samarbeid, ved å legge vekt på kollektivitet fremfor individualisme, slik at gruppen internt bygger personlige forhold fremfor upersonlige forhold.
			
			\item Konstruktiv konflikthåndtering hvor det legges særlig vekt på skillet mellom (a) intellektuelle konflikter knyttet til beslutninger og læring, og (b) interessekonflikter.
			
			\item Lære og internalisere fellesskapsorienterte og demokratiske verdier.
		\end{enumerate}
	
	\section{Kilder til variasjon}
		Tre store kilder til individuell variasjon er å finne i (1) demografisk karakteristikk, (2) personlig karakteristikk, (3) evner/kunnskaper. 
		\textbf{Demografisk variasjon} inkluderer kultur, etnisitet, språk, kjønn, funksjonshemninger, alder, sosial status og geografisk tilhørighet. Med \textbf{personlig karakteristikk} menes ting som økonomisk bakgrunn og kommunikasjonsstil\footnote{For min egen del tenker jeg en kan legge til: generelt humør, humor, interesser, seksuell legning, holding til andre, etc.}. Det nevnes også ting som introvert vs ekstrovert personlighet, en persons tilnærming til problemer; tilfeldig eller prosedyreorientert. Holdinger til innovasjon trekkes også frem her, knyttet til en persons utdannelsesnivå. \textbf{Evner og talenter}, av både sosial og teknisk karakter, er individuelt forskjellige blant alle i gruppen. Det er vanskelig å finne en fungerende/produktiv gruppe der det ikke er stor variasjon i evner og kunnskaper mellom medlemmene.
		
	\section{Viktigheten av å forvalte variasjon}
		Å utnytte variasjon på en slik måte at mange positive og få negative utfall følger, er en av det moderne samfunns største utfordringer. Gode måter å håndtere variasjon på er viktig av flere grunner. 
		\newline \newline
		For det første lever vi i stadig større grad i "en verden". Problemer individer, samfunn og nasjoner står overfor må løses med globalt samarbeid\footnote{Å, kom igjen, Amerika! At jeg mangler tannkrem er ikke en sak for FNs sikkerhetsråd!}. Jo mer gjensidig/internt avhenging verden blir, desto større variasjon innen en gruppe må en forvente. En model som bygger på dette kalles "global village".
		\newline \newline
		For det andre er denne variasjonen praktisk talt umulig å unngå. I den moderne tid er emigrasjon en ny faktor som før ikke på langt nær spilte den samme rollen. I tillegg er det nå mye større kommunikasjon på tvers av sosiale skiller enn det var før.
		\newline \newline
		For det tredje er globaliseringen av verdensøkonomien en stor bidragsyter til å forsterke variasjon: folk flytter på seg for å følge arbeid og karriere, og kommer med det inn i nye grupper der de selv er "variasjonen". 
		
		\begin{quote} Morgendagens effektive grupper (inklusive store grupper som organisasjoner og nasjoner) er de som har lært seg å være produktive med et stort mangfold i medlemskapet. \end{quote}
		
		Resten av dette kapittelet handler om hvordan grupper kan utnytte de positive sidene av mangfoldet, og minimere de negative.
		
	\section{Verdien av mangfold}
		\subsection{Gruppesammensetning og oppgaveutførelse}
			Det er forsket på graden av homogenitet-heterogenitet innen gruppemedlemmers demografiske, personlige og kunnskapsrelaterte attributter. Dette er gjort ved å gi en gruppe tre forskjellige typer oppgaver og måle utfall. (1) Ytelse ved grunnleggende og veldefinerte produksjonsoppgaver, (2) ytelse ved kognitive og intellektuelle oppgaver, (3) ytelse ved kreative og beslutningsorienterte oppgaver.
			
			\subsubsection{Produksjonsoppgaver}
				Produksjonsoppgaver har objektive standarder for evaluering av ytelse, og krever bruk av motoriske og sensoriske evner. Det er funnet at grupper med heterogen sammensetning av tekniske evner yter bedre enn grupper med veldig lik sammensetning, hva gjelder enkle produksjonsoppgaver. 
				\newline \newline
				Det ble funnet at vitenskapsfolk og ingeniører som hadde mye kontakt med kolleger med en annen spesialitet/andre fagfokus hadde høyere produktivitet enn de som ikke hadde denne kontakten\footnote{Det er igjen litt blanding av kort her, etter min mening: en ingeniør gjør vel sjelden enkle produksjonsoppgaver, så det er ikke helt sikkert at ting er sortert riktig i artikkelen...}. Det nevnes også at B-29 bombe-team er flinkere desto mer variasjon det er i gruppemedlemmenes evner. Det samme gjelder for lagidrett.
				
			\subsubsection{Kognitive oppgaver}
				Her er det snakk om problemløsningsoppgaver med fasit (det eksisterer et rett svar). Det er funnet en svak men eksisterende korrelasjon mellom en gruppes ytelse og hvorvidt gruppen er satt sammen av samme eller forskjellige kjønn (forskjellig = bra). 
				\newline \newline
				Grupper satt sammen av medlemmer med forskjellige attributter presterer bedre enn homogene grupper, blant annet fordi i en blandet gruppe er det mer sannsynlig at minst ett av medlemmene har vet hva som er rett svar\footnote{Igjen er det noe muffens med måten ting vektlegges på: jeg tipper det er langt mellom Gløsinger som ikke vet hva svaret på et vanlig sinus-integral skal være; siden det ikke står noe om hvordan utvalgene i undersøkelsene er gjort er det ikke godt å si helt hva som menes med en blandet gruppe.}. 
				
			\subsubsection{Beslutningsoppgaver}
				Denne typen oppgaver kretser om å nå et kompromiss angående et problem der det ikke eksisterer en objektiv/kjent fasit. Undersøkelser viser at en heterogen gruppe gjør bedre, mer kreative valg enn homogene grupper. Det er også funnet at åpne grupper\footnote{Jeg tror her det menes grupper der medlemskapet er flytende i den forstand at nye medlemmer kommer og går.} er mer kreative enn lukkede. En studie viste at etter tre år er en gruppe som opprinnelig var heterogen med tanke på perspektiver og løsningsstrategier for problemer blitt homogen. 				
				\newline \newline
				En studie av flere "management teams" i bankvirksomheter i USA fant at grupper der arbeidserfaringen blant medlemmene var variert gjorde mer innovative valg oftere enn homogene grupper. En annen studie viser at når en sammenlikner (a) en gruppe med varierende evner (abilities) med (b) et enkeltindivid med evner tilsvarende den beste i gruppe (a), presterer gruppa bedre enn individet. Tanken er at den beste i gruppa får noen å "spille ball med" som gjør at nye perspektiver på gammel kunnskap oppstår. 
				
				\begin{center}
					\begin{table}
						\begin{tabular}{l | p{5cm} | p{5cm} }
							\hline
							\textbf{Typer utfall} & \textbf{Personlige attributter} & \textbf{Evner og kunnskaper} \\
							\hline
							Produksjonsoppgaver & Få studier på området, og de finner motstridende resultater. & De få studiene som er gjort tyder på at heterogenitet øker produktiviteten. \\ \hline
							Kognitive oppgaver & Ikke not studier til å trekke en endelig konklusjon, men det er funnet at blandet-kjønn-grupper yter bedre enn samme-kjønn-grupper. & Ingen direkte relevant forskning på denne betydningen. \\ \hline
							Beslutningsoppgaver & Heterogene grupper yter bedre enn homogene. & Heterogenitet i evne-nivå er gunstig. \\ \hline
							Gruppetilhørighet & Heterogene grupper har mindre tilhørighet og større utskiftning. & Ingen relevant forskning. \\ \hline
							Konflikt & Heterogene grupper opplever mer konflikt. & Inger relevant forskning. \\ \hline
						\end{tabular}
						\caption{Gruppekomposisjonens betydning på forskjellige utfall}
						\label{tab:gruppeKomposisjon}
					\end{table}
				\end{center}

		\subsection{Andre utfall}
			Tre viktige momenter som er viktige for en gruppe (utover produktivitet) er (1) fravær, (2) utskiftning og (3) tilfredshet. Disse "verdiene" henger tett sammen med gruppetilhørighet og konfliktnivå.
			\newline			
			
			Det er alt i alt ikke helt sikkert hvordan sammenhengen er mellom gruppens homogenitet og gruppentilhørigheten. Det er derimot noe som tyder på at grad av heterogenitet spiller inn på konfliktnivået. Det ble funnet at demografisk homogenitet reduserte utskiftningen i gruppa, og det er tenkt at dette kan være knyttet til en noe større gruppetilhørighet.
			
			Likhet i innstilling (attitude) ser ut til å være svakt positivt korrelert med gruppetilhørighet. Heterogenitet fører til mer argumentasjon og uenighet. Dette kan ha en positiv effekt på gruppens evne til å løse kompliserte oppgaver.
			
		\subsection{Ulemper ved homogenitet i medlemskap}
			Selv om det kan høres fint ut med en homogen gruppe, har det et knippe ulemper knyttet ved seg. 
			
			For det første vil en homogen gruppe ikke oppleve naturlig indusert uenighet og variasjon i perspektiver, noe som er essensielt for å treffe gode beslutninger i vanskelige saker/prosjekter.
			
			For det andre har slike grupper en tendens til å være risikoaverse, noe som kan bety at de går glipp av muligheter til å øke produktiviteten.
			
			For det tredje vil de i mye større grad inngå i gruppetenking (groupthink\footnote{Jeg vet ikke hva de mener her. Det er en referanse til ett eller annet obskurt verk fra 72 som jeg ikke har giddet å følge opp...}). 
			
			For det fjerde vil en slik gruppe normalt fungere best i en statisk funksjon, mens under skiftende forhold har en homogen gruppe vanskeligere for å yte godt.
			
			På den annen side er det ikke slik at en heterogen gruppe på magisk vis blir perfekt. Å bringe sammen forskjellige mennesker med mangfoldige attributter er bare første skritt. Det skapes en initiell kontakt, men denne kan fort oppleves som vanskelig ("interaction strain" er begrepet på engelsk). Individer føler ubehag og usikkerhet knyttet til hvordan de skal oppføre seg. Dette skaper atypisk oppførsel som overvennligehet etterfulgt av tilbaketrukkenhet.
			
			Under veldig konkurransepregede omgivelser kan en heterogen gruppe få problemer knyttet til oppblomstring av egosentrisitet, defensiv interaksjon og avvisning av ny informasjon.
			
		\subsection{Konklusjoner}
			Siden adferdsforskning er meget komplisert, og forskningen på dette området er mindre omfangsrik er det vanskelig å bruke resultatene her som noe annet enn generelle trekk. 
			
	\section{Barriere mot interaksjon med andre (diverse peers)}
		\subsection{Stereotypier}
			I moderne bruk av ordet er "stereotypi" definert som en forståelse av andre som setter hele grupper av mennesker i samme bås basert på enkelte trekk. Stereotypier (1) er kognitive, (2) reflekterer en sammensatt tro/forståelse fremfor enkelt-biter av informasjon, (3) beskriver attributter, personligheter og karakterer slik at grupper kan bli sammenliknet og evaluert, (4) er delt mellom de som holer samme stereotypi.
			
			Folk skaper seg stereotypier på to måter. Først kategoriserer de informasjon ved å gruppere enkeltting sammen, for å slippe å tenke på alle tingene separat. Deretter differensieres det mellom "inngrupper" og "utgrupper". Folk vil vanligvis anta at medlemmer i en utgruppe er temmelig like, mens de samtidig tenker at medlemmer i den inngruppa de selv ser seg som en del av er veldig forskjellige: en hvit mann kan tenke at alle "hispanics" er like, mens en annen hvit mann, med mange "hispanic" venner ikke ser særlig stor likhet mellom folk fra Puerto Rico, Mexico og Argentina. 
			
			Stereotypier brukes for å maksimere læring av nyttig informasjon\footnote{I kompendiet er det her brukt et begrep, "functionally accurate" som ikke er forklart nærmere.} ved bruk av så lite tid investert i saken som mulig.
			
			Selv om stereotypier er effektive og verdifulle, kan de i en del sammenhenger være til skade. I ekstreme tilfeller kan det som først var en god effekt (spare tid, men fortsatt få informasjon om andre) bli et stort problem: en kan basere seg så utelukkende på stereotypier at en stopper å interagere med nye mennesker. De kan også feste seg mer kulturelt: afro-amerikanere er gode i sport, asiater jobber hardere, menn er mer konkurrerende enn kvinner. 
			
			Folk som holder seg sterkt til stereotypier har en tendens til å begå "fundamental attribusjonsfeil". Dette betyr at de forstår negativ oppførsel fra en minoritet (i gruppa, ikke nødvendigvis kulturell/etnisk) som en konsekvens av predisponert karakteristikk, og positiv oppførsel som resultat av situasjonen. På den annen side tenker de helt motsatt om sin egen oppførsel/prestasjon: når noe går galt er det "de andres" feil; når noe går bra er det min egen fortjeneste.
			
			Stereotypier består og beskyttes på fire forskjellige måter.
			\begin{enumerate}
				\item Stereotypier påvirker hva vi oppfatter og husker om handlingene til et utgruppe-medlem. Våre fordommer gjør at vi husker bedre de tingene som forsterker vårt forutinntatte inntrykk, og glemmer eksempler på det motsatte. 
				\item Man har en tendens til å legge mye større vekt på forskjeller mellom to utgrupper enn på forskjeller innad i utgruppene. I sammenheng med dette vil en ofte legge vekt på forskjellene og utføre handlinger som diskriminerer til fordel for sin egen inngruppe.
				\item Siden en utgruppe er oppfattet som homogen er det vanlig at en tenker at handlinger begått av et individ i gruppa er generaliserbare til alle individer i gruppa. En ol' timer ser en kid kjøre som en tulling, og tenker at alle kidz kjører som tullinger.
				\item "Scapegoating" kan raskt komme som en følge av sterke stereotypier. En uskyldig men også forsvarsløs gruppe blir brukt som svarteper av en større gruppe, som har behov for et utløp for sinne og frustrasjon over en situasjon. Stereotypier om spesielle utgrupper kan skape en "evig svarteper" som alltid får skylden for ting. Les: jøder og nazister.
			\end{enumerate}
			
			Folk som utsettes for en stereotypi kan, dersom de tror på den, komme til å modifisere sin adferd for å passe stereotypien siden de tross alt ønsker å høre til den inngruppa de identifiserer seg med (som en annen gruppe ser på som en stereotyp utgruppe\footnote{Inception, inception}).
			
			Den eneste måten å motvirke en stereotypi er å opparbeide seg mye personlig kunnskap om medlemmer av det en tidligere oppfattet som en homogen utgruppe. Med tiden kan denne informasjonen endre den opprinnelig negative stereotypien til mer nyansert og positiv.
			
		\subsection{Fordommer og diskriminering}
			Fordommer er stereotypier tatt til det ekstreme. Fordommer er bedømminger om andre (individer) basert utelukkende på deres tilhørighet til en annen gruppe en ens egen, og de brukes til å underbygge et overlegen/underlegenhets system.
			
			Typiske fordommer er "sexisme", rasisme, etnosetrisme og aldersisme (..?). Selv de som gjør sitt beste for ikke å være fordømmende kan falle i fella, og dette tyder på at en del slike ting er kulturelt betinget/innlært, og at vi fortsatt har en vei å gå for å bekjempe denne typen tanker.
			
			Følgende forslås for å redusere ens egne fordommer og stereotypier:
			\begin{enumerate}
				\item Innrøm at du har fordommer og forplikt deg til å endre dem.
				\item Identifiser stereotypiene som ligger til grunn for dine fordommer og endre dem.
				\item Identifiser handlinger som følger av dine fordommer og endre dem.
				\item Søk tilbakemelding fra "diverse" venner og kolleger om hvor god du er på å fremme og vise respekt for mangfold.
			\end{enumerate}
			
		\subsection{"Offeret har skylda" og attribusjonsteori}
			"Offeret har skylda"-tankegang går på å tenke at folk gjorde noe for å fortjene det som skjedde med dem: jenta hadde for kort kjole, ransofferet var dumt som gikk med masse cash i veska, etc. Selv offeret selv kan i mange tilfeller tenke at de fikk som fortjent. 
			
			Mye slik tankegang er knyttet til at vi som mennesker har behov for å tro at verden er rettferdig: så lenge vi ikke gjør noe galt kan vi være sikre på at gale ting ikke skjer med oss. Derfor rasjonaliserer man overgrep av forskjellige slag på et slik vis at offeret i alle fall får deler av skylden, slik at vi kan føle oss trygge på at vi ikke kommer til å oppleve det samme, så lenge vi ikke gjør det offeret gjorde for å havne i trøbbel.
			
			Attribusjon er noe vi gjør for å forklare hvorfor noe har inntruffet. Kausal attribusjon deles i to typer: intern og ekstern attribusjon. Når det går bra på en prøve kan vi enten internt attribuere det til at vi er supersmarte, eller ekstern attribuere det til at prøven var veldig enkel. Når det går dårlig kan vi internt attribuere det at vi er dumme, eller eksternt attribuere det til at prøven var veldig vanskelig. En vil normalt internattribuere positive utfall og eksternattribuere negative, slik at vi kan beskytte oss fra følelser som at vi er dumme, verdiløse eller noe annet som er dårlig for selvfølelsen. Det samme går igjen i gruppearbeid: enkeltmedlemmene i gruppa vil ha en tendens til å ta æren for ting som går bra, mens de vil skylde på hele resten av gruppa for ting som går dårlig.
			
		\subsection{Kulturkrasj}
			Til sist nevnes kulturkrasj. Dette er en konflikt knyttet til grunnleggende verdier som oppstår mellom individer av forskjellig kultur. Den mest normale typen kommer av at minoritetsgrupper stiller spørsmål ved majoritetens verdier. Vanlige reaksjoner på dette fra medlemmer av majoriteten er:
			\begin{enumerate}
				\item \textit{Truet}: Reaksjoner inkluderer tilbaketrukkenhet, defensiv holdning og fornektelse.
				\item \textit{Forvirret}: Reaksjoner går på å søke mer informasjon slik at problemet kan forstås i nytt lys eller redefineres.
				\item \textit{Forbedret}: Reaksjoner går på økt forventning, bevissthet og positive handlinger ment å løse problemet.
			\end{enumerate}
			
			Brukt på riktig vis vil kulturkrasj kunne brukes som springbrett for å bedre seg selv, sin kultur og sin forståelse for andres kultur.
			
	\section{Å gjøre medlemsmangfold til en styrke}
		For at en gruppe skal kunne utnytte mangfoldet må medlemmene:
		\begin{enumerate}
			\item Forsikre at en høy grad av positiv avhengighet eksisterer mellom medlemmer
			\item Skape en overordnet gruppeidentitet som (a) forener de varierende personlige identitetene i gruppa, og (b) er basert på pluralistiske verdier.
			\item Forstå forskjellene mellom medlemmer ved å skape personlige bånd/relasjoner som legger til rette for oppriktige diskusjoner.
			\item Rette opp i feilkommunikasjon av alle slag.
		\end{enumerate}
		
		\subsection{Skape overordnet gruppeidentitet}
			Å skape en enhetlig identitet fra mange individuelle personligheter gjøres i fire steg.
			
			\subsubsection{Individuell verdsetting av egen bakgrunn og personlige karakteristikker}
				
				Medlemmer må sette pris på og se sin egen kultur, historie og etnisitet som del av sin identitet\footnote{Her er jeg av forskjellige årsaker dypt uenig; muligens noe vi kan diskutere -- jeg mener etnisitet/nasjonalitet er irrelevant for identitet i realiteten.}. En personlig identitet er et sett konsistente holdninger som definerer "selvet". "Hovedidentiteten" inkorporerer mange underidentiteter slik som kjønnsidentitet, kulturell identitet, etnisk og religiøs identitet, etc. Å respektere sine mange underidentiteter i syntesen av hovedidentiteten sin er grunnlaget for selvrespekt.
				
			\subsubsection{Å verdsette de andre medlemmenes bakgrunner}
				
				En må være var mot å utvikle en identitet som gjør det vanskelig å respektere andres. Særlig vil dette være knyttet til religion og etnisitet. I hvilken grad et gruppemedlems identitet fører til respekt og verdsetting andres mangfold avhenger av å sette sammen en overordnet identitet som tar opp i seg alle medlemmers arv (heritage)\footnote{Litt rart her: å skape en overordnet gruppeidentitet avhenger av at gruppa klarer å skape en overordnet gruppeidentitet\dots}.
				
			\subsubsection{Tilrettelegge for at medlemmer faktisk føler seg som medlemmer av gruppa}
			
				Gruppen vil ha en slags kombinasjonskultur basert på alle medlemmenes individuelle kulturelle bakgrunner. Medlemmene må lære seg å legge vekt på denne overordnede identiteten når konflikter på grunn av individuelle forskjeller skal løses.
				
			\subsubsection{Opprettelse av en slags sosial kontrakt}
			
				Gruppens medlemmer må adoptere et pluralistisk verdigrunnlag basert på demokrati, frihet, likhet, rettferd og individuelle rettigheter\footnote{Vi slenger inn brorskap her også, gjør vi ikke? :-)}. Alle medlemmer får si sitt om hvordan gruppen skal fungere, alle medlemmer har like stor verdi, alle medlemmer har rett til og ansvar for å bidra med sine resurser til gruppens endelige mål. Alle medlemmer har rett til å forvente at gruppen skal forsøke å legge til rette for individets behov. Alle medlemmer må være forberedt på å legge til side sine egne behov i noen situasjoner til fordel for gruppens beste. 
				
				Tanken er å skape en sammenflettet enhet der alle føler de både hører hjemme og føler de har bidratt i utformingen av felles verdier.
				
		\subsection{Å skape intergrupperelasjoner (relasjoner mellom medlemmer)}
			For å bli "sofistikert" og trenet i å relatere deg til, samarbeide med og skape vennskap med forskjellige like (peers), må du:
			\begin{enumerate}
				\item \textit{Faktisk interaksjon:} Oppsøke situasjoner der du kan interagere med mange ulike mennesker fordi du verdsetter bred kulturell forståelse og setter pris på mangfold og viktigheten av dette.
				\item \textit{Tillit:} Bygge tillit ved å være åpen om deg selv og din interesse for å skape krysskulturelle relasjoner. En må også vise at en er til å stole på når andre deler deres mening med deg.
				
				\item \textit{Oppriktighet:} Overbevis de andre medlemmene om å være oppriktige ved åpent å diskutere følelser, tanker og reaksjoner. For å få en komplett forståelse av hva som sårer og fungerer dårlig når en kommuniserer må alle være helt åpne på hva som gjør nettopp dette, slik at ting kan forbedres.
			\end{enumerate}
			
		\subsection{Rette opp i misskommunikasjon}
			Kommunikasjon er en av de mest kompliserte elementene i det å opprettholde et godt forhold til forskjellige medmennesker. For å kommunisere effektivt og godt med mennesker av ulik bakgrunn enn deg selv må du øke din:
			
			\begin{enumerate}
				\item \textit{Språksensitivitet:} Kunnskap om ord og uttrykk og deres korrekthet ved bruk i kommunikasjon med ulike grupper er viktig for å unngå å forsterke stereotypier og for å hindre uklar kommunikasjon.
				\item \textit{Bevisthet angående stilistiske elementer i kommunikasjon:} En må kjenne de grunnleggende elementene i kommunikasjonsstil for forskjellige (relevante) kulturer. Uten disse nyansene er sjansen for dårlig kommunikasjon mye større enn dersom en har forståelse for dette.
			\end{enumerate}
			
			En kan aldri være sikker på at andre ikke missforstår det en prøver å si. Men noen grep kan tas for å redusere sjansen for at det skjer:
			\begin{enumerate}
				\item Ta tak i og diskuter ting med en gang det ser ut til at noen har missforstått noe.
				\item Bruk inkluderende termer, som "deltakere, medlemmer, kvinner, menn" fremfor ekskluderende.
				\item Unngå adjektiv som legger vekt på et tilsynelatende unntak: muslimsk lege, kvinnelig pilot, gammel lærer.
				\item Ved bruk av sitater, referanser, analogier og metaforer, bruk stoff fra flere forskjellige kilder: både Afrika, Asia og Europa (vesten).
				\item Unngå termer ment å devaluere, definere eller latterliggjøre andre, slik som "tilbakestående, gutt, oppvilger".
				\item Vær klar over ords genealogi (opprinnelse), og at dette kan bety mye for noens forståelse av ordet. Med dette menes hva grupper kan legge av undertoner/ladning til et enkelt ord. Eksempel: tidligere var det greit å si "lady", mens nå er det ikke lenger det, da uttrykket for mange kvinner ikke viser forståelse for likestilling og likeverd i samfunnet (det er nedverdigende).
			\end{enumerate}
			
	\section{Sammendrag}
		Sammendragskapittelet gidder jeg ikke å lage et sammendrag av\dots Så dette er enden på visa.
\end{document}